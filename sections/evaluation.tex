\section{Test-bed Environment and Experiments}
\label{sec:evaluation}

About 2 pages that:

\begin{description}
\item[Explains] how the prototype has been tested the test-bed environment.

\item[Explains] what experiments have been done and the results.

\end{description}

\subsection{Frontend testing}
When developing the frontend of our application, multiple testing strategies were performed to make sure that the 
application's functionalities were behaving as expected. One of the advantages of using Angular, is that it provides 
the option to do real-time compilation of the application. This feature made it possible to view the results immediately 
after code changes. This shortened the time spent on troubleshooting errors. The inspect tool in the browser were 
also a great tool that quickly let us inspect the applications. The console feature were handy when inspecting the 
HTML and CSS, and the Network analysis tool gave us a great insight into how the frontend and the backend 
communicated. Alerts were also created to notify the user whenever errors and issues occurred. 
of errors 

