\clearpage
\section{Conclusions}

\noindent The overall design framework for the FeedApp project was pre-established at the beginning of the assignment. This design was further amplified through the course of various lectures that looked into different aspects of the project.  Given our initial unfamiliarity with many of the technologies outlined in the project's scope, we proceeded to adopt those specified in the project guidelines.\\

\noindent The project's brief did offer the option to incorporate one additional technology, which, while seemingly providing a degree of flexibility, also imposed certain constraints on our technological explorations. For example, questions such as the feasibility of implementing the project in Python, the use of a Python persistence model, or Python's capability to interface effectively with a database, remained unexplored. These potential avenues of inquiry could have provided a valabue alternative allowong a more complex design and quicker development approach.\\

\noindent However, the intensive nature of the assignments precluded a thorough investigation into these alternative technological possibilities. As a result, the project proceeded within the confines of the pre-specified technology stack, leaving some questions about potential alternative implementations and technologies unanswered.

\section*{References}
\begin{enumerate}
  	\item Angular Architecture Guide: \url{https://angular.io/guide/architecture }
	\item Spring Official Documentation \url{https://docs.spring.io/spring-framework/reference/index.html}
          \item JSON Web Tokens  \url{https://datatracker.ietf.org/doc/html/rfc7519/}
 	\item Hibernate ORM \url{https://hibernate.org/orm/}
	\item Mosquitto MQTT \url{https://mosquitto.org}
	\item Arduino \url{https://www.arduino.cc/reference/en/}
  	\item ESP8266 for IoT \url{https://www.nabto.com/esp8266-for-iot-complete-guide/}
	\item Rapid Software Testing  \url{https://rapid-software-testing.com/}
\end{enumerate}