\section{Software Technology Stack}
\label{sec:technology}

\textcolor{red}{Introduce in (sufficient) depth the key concepts and architecture of the chosen software technologies. As part if this, you may consider using a running example to introduce the technology.}

\textcolor{red}{Emphasize the “new” software technologies that was selected by the group and which has not been covered in the course.}

\textcolor{red}{This part and other parts of the report probably needs to refer to
figures. Figure~\ref{fig:framework} from \cite{brown:96} just
illustrates how figure can be included in the report.}

Angular:
The front-end of the FeedApp is developed using Angular. Angular is a widely used framework that is used for building single-page applications (SPAs).

Fundamental Consepts of Angular: 
\begin{description}
\item[Components and Views]: The angular application is built upon components. For every component, there is a HTML template for the content of the web interface and a TypeScript class that controls the logic. The components are what defines the views, which is what  is displayed to the user on the webpage. 
\item[Dependency Injection (DI)]: Angular´s dependency injection (DI) system provides services to components. Services are classes that can contain business logic, data handling and functionalities that can be useful in multiple components. In our FeedApp implementation, we created services dedicated to managing authentication and handling poll data. These services made it possible to streamline data interactions and logic across different components, making it easier to control that the underlying operations and management of the data were handled consistently. 
\item[Routing]: The Angular Router handles the navigation between different views as users performs different tasks. This is a key element in SPAs since instead of reloading the page

Resourses used for writing this paragraph: https://angular.io/guide/architecture 
\end{description}

Spring Boot 

JSON Web Tokens 

H2, Hibernate 

Java Persistence API 

Mosquitto MQTT

\begin{figure}
  \centering
  \includegraphics[scale=0.5]{figs/framework.png}
  \caption{Software technology evaluation framework.}
  \label{fig:framework}
\end{figure}
